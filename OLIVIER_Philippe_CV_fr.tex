\documentclass[letterpaper,MMMyyyy,nonstopmode]{simpleresumecv}

% Source of this command: tex.stackexchange.com/a/294990
\usepackage{tikz}
\newcommand{\ExternalLink}{
  \tikz[x=1.2ex, y=1.2ex, baseline=-0.05ex]{
    \begin{scope}[x=1ex, y=1ex]
      \clip (-0.1,-0.1) 
      --++ (-0, 1.2) 
      --++ (0.6, 0) 
      --++ (0, -0.6) 
      --++ (0.6, 0) 
      --++ (0, -1);
      \path[draw, 
        line width = 0.5, 
        rounded corners=0.5] 
      (0,0) rectangle (1,1);
    \end{scope}
    \path[draw, line width = 0.5] (0.5, 0.5) 
    -- (1, 1);
    \path[draw, line width = 0.5] (0.6, 1) 
    -- (1, 1) -- (1, 0.6);
  }
}

\newcommand{\CVAuthor}{Philippe Olivier}
\newcommand{\CVWebpage}{github.com/PhilippeOlivier}

\hypersetup{
pdftitle={\CVTitle},
pdfauthor={\CVAuthor},
pdfsubject={\CVWebpage},
pdfcreator={XeLaTeX},
pdfproducer={},
pdfkeywords={},
unicode=true,
bookmarks=false,
bookmarksopen=false,
pdfstartview=FitH,
pdfpagelayout=OneColumn,
pdfpagemode=UseOutlines,
hidelinks,
breaklinks}

\newcommand{\Code}[1]{\mbox{\textbf{#1}}}
\newcommand{\CodeCommand}[1]{\mbox{\textbf{\textbackslash{#1}}}}
\pagenumbering{gobble}

\begin{document}

%%%%%%%%%
% TITRE %
%%%%%%%%%

\Title{\CVAuthor}

\begin{SubTitle}
\href{mailto:philippe.olivier@polymtl.ca}
{philippe.olivier@polymtl.ca}
\,\SubBulletSymbol\,
514.433.5700
\,\SubBulletSymbol\,
\href{https://www.github.com/PhilippeOlivier}
{\url{\CVWebpage}\ExternalLink}
\end{SubTitle}

\begin{Body}

%%%%%%%%%%%%%%%
%% ÉDUCATION %%
%%%%%%%%%%%%%%%

\Section
{Éducation}{}{}

\Entry
\textbf{Polytechnique Montréal}

Doctorat en génie informatique
\hfill
Août 2016 -- Présent

\Gap
\BulletItem
Intérêts de recherche
\begin{Detail}
\SubBulletItem
Optimisation combinatoire
\SubBulletItem
Programmation par contraintes
\SubBulletItem
Programmation en nombres entiers
\end{Detail}

\Gap
\BulletItem
Directeurs de recherche : Gilles Pesant et Andrea Lodi

\BigGap
\Entry
\textbf{Université Laval}

Baccalauréat en informatique
\hfill
Août 2012 -- Mai 2016

%%%%%%%%%%%%%%%%%%%%%%%%%%%%
%% EXPÉRENCE DE RECHERCHE %%
%%%%%%%%%%%%%%%%%%%%%%%%%%%%

\Section
{Expérience de recherche}{}{}

\Entry
\textbf{Publications}

% TODO: Add url for CPAIOR 2018 (paper).
\Gap\Gap
Olivier, P., Lodi, A., and Pesant, G. (2018) "A Comparison of Optimization Methods for Multi-Objective Constrained Bin Packing Problems". In \emph{Integration of AI and OR Techniques in Constraint Programming, Delft, Netherlands, (CPAIOR 2018)}.

\Gap\Gap
\Entry
\textbf{Présentations en conférences}

% TODO: Add url for CPAIOR 2018.
\Gap\Gap
CPAIOR 2018 (Delft, Pays-Bas)
\hfill
Juin 2018
\begin{Detail}A Comparison of Optimization Methods for Multi-Objective Constrained Bin Packing Problems\end{Detail}

% TODO: Add url for JOPT 2018.
\Gap\Gap
JOPT 2018 (Montréal, Canada)
\hfill
Mai 2018
\begin{Detail}A Comparison of Optimization Methods for Multi-Objective Constrained Bin Packing Problems\end{Detail}

\Gap\Gap
IFORS 2017 (Québec, Canada)
\hfill
Juil. 2017
\begin{Detail}\href{https://www.euro-online.org/conf/ifors2017/treat_abstract?paperid=1523}{Solving the Wedding Seating Problem by Constraint Programming\ExternalLink}\end{Detail}

\Gap\Gap
\Entry
\textbf{Membre}

\Gap\Gap
\href{http://www.polymtl.ca/labo-quosseca/}
{Laboratoire Quosséça\ExternalLink}
\hfill
Depuis 2016

\Gap\Gap
\href{http://cerc-datascience.polymtl.ca}
{Chaire d'excellence en recherche du Canada sur la science des données pour la prise de décision en temps réel\ExternalLink}
\hfill
Depuis 2016

%%%%%%%%%%%%%%%%%%%%%%%%%%
%% IMPLICATION SCOLAIRE %%
%%%%%%%%%%%%%%%%%%%%%%%%%%

\Section
{Implication scolaire}{}{}

\Entry
\textbf{Polytechnique Montréal}

Chargé de laboratoire
\BulletItem
INF4705 : Analyse et conception d'algorithmes
\hfill
Jan. 2018 -- Présent
\small \\ Notation asymptotique, classes de complexité, patrons de conception d'algorithmes, métaheuristiques. \normalsize

\Gap\Gap
\Entry
\textbf{Université Laval}

\Gap\Gap
\href{https://www.ift.ulaval.ca/vie-etudiante/prix-pierre-ardouin/}{Prix Pierre Ardouin\ExternalLink}
\hfill Hiver 2014
\small \\ Meilleur projet en génie logiciel orienté-objet. \normalsize

\Gap\Gap
Tuteur en informatique et mathématiques
\hfill Sept. 2013 -- Déc. 2013

%%%%%%%%%%%%%%%%%%%%%%%%%%%
%% EXPÉRIENCE DE TRAVAIL %%
%%%%%%%%%%%%%%%%%%%%%%%%%%%

\Section
{Expérience \quad\quad de travail}{}{}

\Entry
\textbf{Hydro-Québec TransÉnergie}

Stagiaire en informatique
\hfill
Sept. 2014 -- Avr. 2015
\small \\ Automatisation du transfert de données entre une base de données et un logiciel spécialisé. Automatisation d'une partie des tests effectués par des ingénieurs électrique.\normalsize

%%%%%%%%%%%%%%%%%%%
%% CONTRIBUTIONS %%
%%%%%%%%%%%%%%%%%%%

%% \Section
%% {Contributions}{}{}

%% \Entry
%% \textbf{CSPLib: A problem library for constraints}

%% Problem XXX: The Wedding Seating Problem
%% \hfill
%% DATE

%%%%%%%%%%%%%%%%%%%%%%%%%%%%
%% COMPÉTENCES TECHNIQUES %%
%%%%%%%%%%%%%%%%%%%%%%%%%%%%

\Section
{Compétences techniques}{}{}

C++, Java, Python, CPLEX Optimizer, CPLEX CP Optimizer (ILOG), MiniZinc, Bash, Git, Linux, \LaTeX.

\end{Body}

\end{document}
