% Many parts taken from https://github.com/zachscrivena/simple-resume-cv

\documentclass{memoir}
\usepackage[T1]{fontenc}
\usepackage[utf8]{inputenc}
\usepackage{hyperref}
\usepackage[margin=1in]{geometry}

\begin{document}
\pagenumbering{gobble}
\reversemarginpar

\newcommand{\UseHeadingFont}{\normalfont}
\newcommand{\UseSectionFont}{\UseHeadingFont\fontsize{9pt}{11pt}\selectfont\bfseries}
\newcommand{\UseSubSectionFont}{\UseHeadingFont\fontsize{8pt}{10pt}\selectfont\bfseries}

\newcommand{\Section}[1]{\bigskip\item[\smash{\parbox[t]{1.2in}{\UseSectionFont\raggedright\MakeUppercase{#1}}}]}
\newcommand{\SubSection}[1]{\par\bigskip{\UseSubSectionFont\raggedright\MakeUppercase{#1}}\par}

\newcommand{\Item}{
  \normalsize
  \smallskip
  \par
  \vspace{0\baselineskip}
  \parshape 1
  \labelwidth
  \linewidth
  \ignorespaces}

\newcommand{\BulletSymbol}{{\normalfont\fontsize{6.5}{7.8}\selectfont\raisebox{0.17em}{\textbullet}}}
\newsavebox{\BulletItemIndentBox}
\newlength{\BulletItemIndentWidth}
\newcommand{\BulletItem}{\par\vspace{0\baselineskip}
  \savebox{\BulletItemIndentBox}{\hspace{1.5mm}\BulletSymbol\hspace{1.5mm}}
  \settowidth{\BulletItemIndentWidth}{\usebox{\BulletItemIndentBox}}
  \parshape 2
  \labelwidth \linewidth
  \dimexpr\labelwidth+\BulletItemIndentWidth\relax\dimexpr\linewidth-\BulletItemIndentWidth\relax
  \usebox{\BulletItemIndentBox}
  \ignorespaces}

\noindent{\LARGE \textsc{Philippe Olivier}}
\hfill
\begin{minipage}{.4\textwidth}
\footnotesize{
  \flushright
  \texttt{\href{mailto:philippe@pedtsr.ca}{philippe@pedtsr.ca}} \\
  \texttt{\href{https://www.github.com/PhilippeOlivier}{github.com/PhilippeOlivier}} \\
  \texttt{1-514-433-5700} \\}
\end{minipage}

\vspace{0.5in}

\par
\vspace{1.5\baselineskip}
\begin{list}{}{
    \setlength\leftmargin{1.2in}
    \setlength\rightmargin{0in}
    \setlength\labelwidth{1.2in}
    \setlength\labelsep{0in}
    \setlength\listparindent{0in}
    \setlength\itemindent{0in}
    \setlength\parskip{0in}
    \setlength\topsep{0in}
    \setlength\parsep{0in}
    \setlength\itemsep{0.90\baselineskip}
    \setlength\partopsep{0in}}

  %%%%%%%%%%%%%%%%%%%%%%%%%%%%%%%%%%%%%%%%%%
  
  \Section{Travail}

  \SubSection{Travail contractuel}

  \Item \textbf{pganalyze} \hfill 08/2022--Présent
  \BulletItem Construction d'un modèle d'optimisation pour la sélection automatique d'index pour les bases de données
  \BulletItem Présentation de ce modèle aux conférences JOPT 2023 et PGCon 2023

  \Item \textbf{Integrated Reasoning} \hfill 06/2023--Présent
  \BulletItem Conseiller scientifique pour la recherche opérationnelle, la modélisation, et l'optimisation
  
  %%%%%%%%%%%%%%%%%%%%%

  \SubSection{Emplois}

  \Item \textbf{Hydro-Québec TransÉnergie} \hfill 09/2014--04/2015
  \BulletItem Automatisation du transfert de données de/vers un logiciel spécialisé
  \BulletItem Automatisation des tests faits par des ingénieurs électriques

  %%%%%%%%%%%%%%%%%%%%%%%%%%%%%%%%%%%%%%%%%%
  
  \Section{Éducation}

  \Item \textbf{Polytechnique Montréal} \hfill 08/2016--05/2021 \\
  Doctorat en génie informatique

  \Item \textbf{Université Laval} \hfill 08/2012--05/2016 \\
  Baccalauréat en informatique

  %%%%%%%%%%%%%%%%%%%%%%%%%%%%%%%%%%%%%%%%%%
  
  \Section{Recherche}

  \SubSection{Intérêts}
  
  \BulletItem Recherche opérationnelle
  \BulletItem Programmation par contraintes
  \BulletItem Programmation en nombres entiers

  %%%%%%%%%%%%%%%%%%%%%
  
  \SubSection{Publications}

  \Item \href{https://arxiv.org/abs/2101.03716}{\textbf{Fairness over Time in Dynamic Resource Allocation with an Application in Healthcare}} \\
  Lodi, A., Olivier, P., Pesant, G., and Sankaranarayanan S. \\
  \emph{Mathematical Programming} (2022)

  \Item \href{https://doi.org/10.1007/s10288-021-00486-x}{\textbf{Measures of Balance in Combinatorial Optimization}} \\
  Olivier, P., Lodi, A., and Pesant, G. \\
  \emph{4OR} (2021)

  \Item \href{https://doi.org/10.1287/ijoc.2020.0983}{\textbf{The Quadratic Multiknapsack Problem with Conflicts and Balance Constraints}} \\
  Olivier, P., Lodi, A., and Pesant, G. \\
  \emph{INFORMS Journal on Computing} (2020)

  \Item \href{https://doi.org/10.1007/978-3-319-93031-2_33}{\textbf{A Comparison of Optimization Methods for Multi-Objective Constrained Bin Packing Problems}} \\
  Olivier, P., Lodi, A., and Pesant, G. \\
  \emph{Integration of AI and OR Techniques in Constraint Programming, Delft, Netherlands, (CPAIOR 2018)} (2018)

  %%%%%%%%%%%%%%%%%%%%%
  \pagebreak % TEMP
  \SubSection{Présentations en conférences}

  \Item \textbf{PGCon 2023 (Ottawa, Canada)} \hfill 06/2023 \\
  \href{https://www.pgcon.org/events/pgcon_2023/schedule/session/422-automating-index-selection-using-constraint-programming/}{Automating Index Selection Using Constraint Programming}

  \Item \textbf{JOPT 2023 (Montréal, Canada)} \hfill 05/2023 \\
  \href{https://symposia.cirrelt.ca/CORS-JOPT/fr/schedule?slot_id=2207}{Optimizing Database Index Selection Using Constraint Programming}
  
  \Item \textbf{CPAIOR 2018 (Delft, Pays-Bas)} \hfill 06/2018 \\
  \href{http://icaps18.icaps-conference.org/schedule}{A Comparison of Optimization Methods for Multi-Objective Constrained Bin Packing Problems}

  \Item \textbf{JOPT 2018 (Montréal, Canada)} \hfill 05/2018 \\
  \href{https://symposia.gerad.ca/jopt2018/en/schedule?slot_id=1374}{A Comparison of Optimization Methods for Multi-Objective Constrained Bin Packing Problems}

  \Item \textbf{IFORS 2017 (Québec, Canada)} \hfill 07/2017 \\
  \href{https://www.euro-online.org/conf/ifors2017/treat_abstract?paperid=1523}{Solving the Wedding Seating Problem by Constraint Programming}

  %%%%%%%%%%%%%%%%%%%%%

  \SubSection{Présentations de posters}

  \Item \textbf{CP 2019 (Stamford, États-Unis)} \hfill 10/2019 \\
  Measures of Balance in Combinatorial Optimization

  %%%%%%%%%%%%%%%%%%%%%

  \SubSection{Membre}

  \Item \href{https://www.polymtl.ca/labo-quosseca/en/members/alumni/doctoral-students}{\textbf{Laboratoire Quosséça}} \hfill 08/2016--05/2021

  \Item \href{http://cerc-datascience.polymtl.ca/person/philippe-olivier}{\textbf{Chaire d'excellence en recherche du Canada} \hfill 08/2016--05/2021 \\ \textbf{sur la science des données pour la prise de décision en temps réel}}
  
  %%%%%%%%%%%%%%%%%%%%%%%%%%%%%%%%%%%%%%%%%%

  \Section{Enseigne-\\ment}

  \SubSection{Chargé de cours}

  \Item \textbf{Polytechnique Montréal}
  \BulletItem INF1005D : Programmation procédurale en Python \hfill 01/2023--05/2023
  \BulletItem INF1005D : Programmation procédurale en Python \hfill 08/2022--12/2022
  \BulletItem INF1005D : Programmation procédurale en Python \hfill 08/2021--12/2021

  \Item \textbf{Université du Québec à Montréal}
  \BulletItem INF1070 : Utilisation et administration des systèmes \hfill 08/2022--12/2022 \\ informatiques (deux cours)
  \BulletItem INF1070 : Utilisation et administration des systèmes informatiques \hfill 01/2022--04/2022

  %%%%%%%%%%%%%%%%%%%%%

  \SubSection{Chargé de laboratoire}

  \Item \textbf{Polytechnique Montréal}
  \BulletItem INF4705/INF8775 : Analyse et conception d'algorithmes \hfill 01/2018--12/2019

  \Item \textbf{Université Laval} \hfill 08/2013--12/2013

  %%%%%%%%%%%%%%%%%%%%%%%%%%%%%%%%%%%%%%%%%%

  \Section{Projets}

  \Item \textbf{Fantasy Solver} \hfill 06/2021--Présent \\
  Solveur à objectifs multiples pour la génération d'équipes optimales dans les tournois à entrées multiples des \emph{Daily Fantasy Sports} (DFS). Il s'agit (à ma connaissance) du seul solveur exact pour la génération d'équipes multiples pour les DFS.
  
\end{list}
\par
\end{document}